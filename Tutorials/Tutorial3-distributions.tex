\documentclass[]{article}
\usepackage{lmodern}
\usepackage{amssymb,amsmath}
\usepackage{ifxetex,ifluatex}
\usepackage{fixltx2e} % provides \textsubscript
\ifnum 0\ifxetex 1\fi\ifluatex 1\fi=0 % if pdftex
  \usepackage[T1]{fontenc}
  \usepackage[utf8]{inputenc}
\else % if luatex or xelatex
  \ifxetex
    \usepackage{mathspec}
  \else
    \usepackage{fontspec}
  \fi
  \defaultfontfeatures{Ligatures=TeX,Scale=MatchLowercase}
\fi
% use upquote if available, for straight quotes in verbatim environments
\IfFileExists{upquote.sty}{\usepackage{upquote}}{}
% use microtype if available
\IfFileExists{microtype.sty}{%
\usepackage{microtype}
\UseMicrotypeSet[protrusion]{basicmath} % disable protrusion for tt fonts
}{}
\usepackage[margin=1in]{geometry}
\usepackage{hyperref}
\hypersetup{unicode=true,
            pdftitle={Tutorial 3 - Circular distributional assumptions},
            pdfauthor={Arjan Huizing},
            pdfborder={0 0 0},
            breaklinks=true}
\urlstyle{same}  % don't use monospace font for urls
\usepackage{color}
\usepackage{fancyvrb}
\newcommand{\VerbBar}{|}
\newcommand{\VERB}{\Verb[commandchars=\\\{\}]}
\DefineVerbatimEnvironment{Highlighting}{Verbatim}{commandchars=\\\{\}}
% Add ',fontsize=\small' for more characters per line
\usepackage{framed}
\definecolor{shadecolor}{RGB}{248,248,248}
\newenvironment{Shaded}{\begin{snugshade}}{\end{snugshade}}
\newcommand{\KeywordTok}[1]{\textcolor[rgb]{0.13,0.29,0.53}{\textbf{#1}}}
\newcommand{\DataTypeTok}[1]{\textcolor[rgb]{0.13,0.29,0.53}{#1}}
\newcommand{\DecValTok}[1]{\textcolor[rgb]{0.00,0.00,0.81}{#1}}
\newcommand{\BaseNTok}[1]{\textcolor[rgb]{0.00,0.00,0.81}{#1}}
\newcommand{\FloatTok}[1]{\textcolor[rgb]{0.00,0.00,0.81}{#1}}
\newcommand{\ConstantTok}[1]{\textcolor[rgb]{0.00,0.00,0.00}{#1}}
\newcommand{\CharTok}[1]{\textcolor[rgb]{0.31,0.60,0.02}{#1}}
\newcommand{\SpecialCharTok}[1]{\textcolor[rgb]{0.00,0.00,0.00}{#1}}
\newcommand{\StringTok}[1]{\textcolor[rgb]{0.31,0.60,0.02}{#1}}
\newcommand{\VerbatimStringTok}[1]{\textcolor[rgb]{0.31,0.60,0.02}{#1}}
\newcommand{\SpecialStringTok}[1]{\textcolor[rgb]{0.31,0.60,0.02}{#1}}
\newcommand{\ImportTok}[1]{#1}
\newcommand{\CommentTok}[1]{\textcolor[rgb]{0.56,0.35,0.01}{\textit{#1}}}
\newcommand{\DocumentationTok}[1]{\textcolor[rgb]{0.56,0.35,0.01}{\textbf{\textit{#1}}}}
\newcommand{\AnnotationTok}[1]{\textcolor[rgb]{0.56,0.35,0.01}{\textbf{\textit{#1}}}}
\newcommand{\CommentVarTok}[1]{\textcolor[rgb]{0.56,0.35,0.01}{\textbf{\textit{#1}}}}
\newcommand{\OtherTok}[1]{\textcolor[rgb]{0.56,0.35,0.01}{#1}}
\newcommand{\FunctionTok}[1]{\textcolor[rgb]{0.00,0.00,0.00}{#1}}
\newcommand{\VariableTok}[1]{\textcolor[rgb]{0.00,0.00,0.00}{#1}}
\newcommand{\ControlFlowTok}[1]{\textcolor[rgb]{0.13,0.29,0.53}{\textbf{#1}}}
\newcommand{\OperatorTok}[1]{\textcolor[rgb]{0.81,0.36,0.00}{\textbf{#1}}}
\newcommand{\BuiltInTok}[1]{#1}
\newcommand{\ExtensionTok}[1]{#1}
\newcommand{\PreprocessorTok}[1]{\textcolor[rgb]{0.56,0.35,0.01}{\textit{#1}}}
\newcommand{\AttributeTok}[1]{\textcolor[rgb]{0.77,0.63,0.00}{#1}}
\newcommand{\RegionMarkerTok}[1]{#1}
\newcommand{\InformationTok}[1]{\textcolor[rgb]{0.56,0.35,0.01}{\textbf{\textit{#1}}}}
\newcommand{\WarningTok}[1]{\textcolor[rgb]{0.56,0.35,0.01}{\textbf{\textit{#1}}}}
\newcommand{\AlertTok}[1]{\textcolor[rgb]{0.94,0.16,0.16}{#1}}
\newcommand{\ErrorTok}[1]{\textcolor[rgb]{0.64,0.00,0.00}{\textbf{#1}}}
\newcommand{\NormalTok}[1]{#1}
\usepackage{graphicx,grffile}
\makeatletter
\def\maxwidth{\ifdim\Gin@nat@width>\linewidth\linewidth\else\Gin@nat@width\fi}
\def\maxheight{\ifdim\Gin@nat@height>\textheight\textheight\else\Gin@nat@height\fi}
\makeatother
% Scale images if necessary, so that they will not overflow the page
% margins by default, and it is still possible to overwrite the defaults
% using explicit options in \includegraphics[width, height, ...]{}
\setkeys{Gin}{width=\maxwidth,height=\maxheight,keepaspectratio}
\IfFileExists{parskip.sty}{%
\usepackage{parskip}
}{% else
\setlength{\parindent}{0pt}
\setlength{\parskip}{6pt plus 2pt minus 1pt}
}
\setlength{\emergencystretch}{3em}  % prevent overfull lines
\providecommand{\tightlist}{%
  \setlength{\itemsep}{0pt}\setlength{\parskip}{0pt}}
\setcounter{secnumdepth}{0}
% Redefines (sub)paragraphs to behave more like sections
\ifx\paragraph\undefined\else
\let\oldparagraph\paragraph
\renewcommand{\paragraph}[1]{\oldparagraph{#1}\mbox{}}
\fi
\ifx\subparagraph\undefined\else
\let\oldsubparagraph\subparagraph
\renewcommand{\subparagraph}[1]{\oldsubparagraph{#1}\mbox{}}
\fi

%%% Use protect on footnotes to avoid problems with footnotes in titles
\let\rmarkdownfootnote\footnote%
\def\footnote{\protect\rmarkdownfootnote}

%%% Change title format to be more compact
\usepackage{titling}

% Create subtitle command for use in maketitle
\newcommand{\subtitle}[1]{
  \posttitle{
    \begin{center}\large#1\end{center}
    }
}

\setlength{\droptitle}{-2em}
  \title{Tutorial 3 - Circular distributional assumptions}
  \pretitle{\vspace{\droptitle}\centering\huge}
  \posttitle{\par}
  \author{Arjan Huizing}
  \preauthor{\centering\large\emph}
  \postauthor{\par}
  \predate{\centering\large\emph}
  \postdate{\par}
  \date{16 April 2018}


\begin{document}
\maketitle

\subsection{Circular distributions}\label{circular-distributions}

It is important to investigate the way that data is distributed. Not
only do many models assume that data has a certain distribution, even
simple descriptives such as the median or mean may paint a incorrect
picture of our data if we take them at face value.

When we analyse circular data it is just as important that we visualize
the distribution of data points. In the previous tutorial we noted it
was possible for mean resultant lengths to indicate a very low
concentration even when in our data this may not be the case. For
example, take the dataset specified below with a mean resultant length
of \(>0.001\). This value would lead us to believe that the data is
spread entirely across the circle.

\begin{Shaded}
\begin{Highlighting}[]
\NormalTok{example <-}\StringTok{ }\KeywordTok{as.circular}\NormalTok{(}\KeywordTok{c}\NormalTok{(}\DecValTok{350}\NormalTok{, }\DecValTok{355}\NormalTok{, }\DecValTok{0}\NormalTok{, }\DecValTok{5}\NormalTok{, }\DecValTok{10}\NormalTok{, }
                         \DecValTok{185}\NormalTok{, }\DecValTok{190}\NormalTok{, }\DecValTok{180}\NormalTok{, }\DecValTok{170}\NormalTok{, }\DecValTok{175}\NormalTok{), }\DataTypeTok{units =} \StringTok{"degrees"}\NormalTok{)}
\KeywordTok{rho.circular}\NormalTok{(example)}
\end{Highlighting}
\end{Shaded}

\begin{verbatim}
## [1] 3.861386e-17
\end{verbatim}

However, when inspecting the actual distribution we can see that there
are two modal groups on opposite ends of the circle.

\includegraphics{Tutorial3-distributions_files/figure-latex/unnamed-chunk-2-1.pdf}

One can also imagine what would happen if we asked for the mean
direction of this dataset. As all data points have an exact opposite
value, the mean will be undefined.

\begin{Shaded}
\begin{Highlighting}[]
\KeywordTok{mean}\NormalTok{(example)}
\end{Highlighting}
\end{Shaded}

\begin{verbatim}
## Circular Data: 
## Type = angles 
## Units = degrees 
## Template = none 
## Modulo = asis 
## Zero = 0 
## Rotation = counter 
## [1] NA
\end{verbatim}

\subsection{Distributional
assumptions}\label{distributional-assumptions}

Many of the statistical models we use also assume the data were obtained
from a certain probability distribution.

\includegraphics{Tutorial3-distributions_files/figure-latex/unnamed-chunk-4-1.pdf}

\subsection{Plotting circular data in
R}\label{plotting-circular-data-in-r}

It is thus always a good idea to plot the dataset before any analysis
takes place. The \texttt{circular} package contains the function
\texttt{plot.circular} for this purpose. An example of this is given
below, using a dataset on flight direction of pigeons. Note how the
argument \(stack = T\) is passed onto the function. The default value
for this is \(FALSE\), which would result in similar values overlapping.

\includegraphics{Tutorial3-distributions_files/figure-latex/unnamed-chunk-5-1.pdf}

It can sometimes also be useful to inspect circular data on a continues
scale. This can be done by transforming the data from circular to
continues using \texttt{as.numeric}. Keep in mind that values that are
outside of the circular range will need to be calculated down to their
remainder using \(%%\).

\begin{Shaded}
\begin{Highlighting}[]
\KeywordTok{hist}\NormalTok{(}\KeywordTok{as.numeric}\NormalTok{(data),}
     \DataTypeTok{breaks =} \DecValTok{360}\NormalTok{,}
     \DataTypeTok{xlim =} \KeywordTok{c}\NormalTok{(}\DecValTok{0}\NormalTok{,}\DecValTok{360}\NormalTok{),}
     \DataTypeTok{main =} \StringTok{"Pigeon flight directions"}\NormalTok{)}
\end{Highlighting}
\end{Shaded}

\includegraphics{Tutorial3-distributions_files/figure-latex/unnamed-chunk-6-1.pdf}

\begin{Shaded}
\begin{Highlighting}[]
\CommentTok{#Testing symmetry using a symmetry plot?}
\end{Highlighting}
\end{Shaded}

\subsection{Uniformity}\label{uniformity}

Uniformally distributed data indicates an even spread of observations
across the entire circular range. This can sometimes be difficult to
observe visually, which is why it is recommended to also assess
uniformity using uniformity plots and statistical tests.

\begin{Shaded}
\begin{Highlighting}[]
\CommentTok{#test for uniformity.}

\CommentTok{#uniformity plot}
\CommentTok{#kuiper.test(data)}
\CommentTok{#rao.test(data)}
\end{Highlighting}
\end{Shaded}


\end{document}
